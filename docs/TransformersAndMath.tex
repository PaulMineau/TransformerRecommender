\documentclass[11pt,a4paper]{article}
\usepackage[utf8]{inputenc}
\usepackage[T1]{fontenc}
\usepackage{amsmath}
\usepackage{amsfonts}
\usepackage{amssymb}
\usepackage{geometry}
\usepackage{hyperref}
\usepackage{fancyhdr}
\usepackage{enumitem}
\usepackage{mathtools}

\geometry{margin=1in}
\pagestyle{fancy}
\fancyhf{}
\rhead{Mathematical Formulation of Transformer Attention}
\lhead{TransformerRecommender Documentation}
\cfoot{\thepage}

\title{\textbf{Mathematical Formulation of Transformer Attention}}
\author{TransformerRecommender Project}
\date{\today}

\begin{document}

\maketitle

\section{Introduction}

Transformer attention uses queries, keys, and values to compute weighted sums of values. Given an input sequence of $N$ tokens represented by row vectors in a matrix $X\in\mathbb{R}^{N\times d_{\text{model}}}$, we first project $X$ into query, key, and value spaces by learned matrices. Concretely, for one attention head we use weight matrices $W^Q,W^K\in\mathbb{R}^{d_{\text{model}}\times d_k}$ and $W^V\in\mathbb{R}^{d_{\text{model}}\times d_v}$ to form:

\begin{itemize}
\item $Q = XW^Q \in \mathbb{R}^{N\times d_k}$ (queries),
\item $K = XW^K \in \mathbb{R}^{N\times d_k}$ (keys),
\item $V = XW^V \in \mathbb{R}^{N\times d_v}$ (values).
\end{itemize}

Each row $q_i$ of $Q$ is the query vector for token $i$, each row $k_j$ of $K$ is a key vector for token $j$, and each row $v_j$ of $V$ is a value vector for token $j$. Intuitively, each query $q_i$ ``asks'' how much attention to pay to each key $k_j$, and these attention weights are used to form a weighted sum of the corresponding values $v_j$. This projection step is summarized by Jurafsky \& Martin (2025):

\begin{quote}
``For one head we multiply $X$ by the query, key, and value matrices $W^Q$, $W^K$, $W^V$ to produce matrices $Q$, $K$, $V$ containing all the key, query, and value vectors: $Q=XW^Q, K=XW^K, V=XW^V$''.
\end{quote}

Throughout, $d_k$ is the dimensionality of the key/query vectors and $d_v$ is the dimensionality of the value vectors. In self-attention typically $N$ (sequence length) equals the number of queries and keys.

\section{Scaled Dot-Product Attention}

The core of the Transformer's attention mechanism is scaled dot-product attention. We compute raw similarity scores between every query and every key by a matrix dot-product. Let 
$$S = QK^\top\in\mathbb{R}^{N\times N},$$
so that $S_{ij}=q_i\cdot k_j$ is the dot product between query $i$ and key $j$. To convert these scores into normalized weights, we apply the following steps:

\subsection{Step 1: Scale}
Divide the score matrix by $\sqrt{d_k}$:
$$S' = \frac{QK^\top}{\sqrt{d_k}}.$$

The factor $1/\sqrt{d_k}$ prevents the dot products from growing too large in magnitude (for high-dimensional $q_i,k_j$), which would make the softmax saturate with very small gradients. Vaswani et al. (2017) note that without this scaling the variance of $q_i\cdot k_j$ grows with $d_k$, so dividing by $\sqrt{d_k}$ keeps values in a more moderate range.

\subsection{Step 2: Softmax}
Apply the softmax function row-wise to $S'$ to obtain attention weights:
$$A = \text{softmax}(S'),\qquad A_{ij} = \frac{\exp(S'_{ij})}{\sum_{j'=1}^N \exp(S'_{ij'})}.$$

This yields a matrix $A\in\mathbb{R}^{N\times N}$ where each row sums to 1. Softmax converts arbitrary scores into a probability distribution over keys for each query. It ensures all weights $A_{ij}\in[0,1]$ and $\sum_j A_{ij}=1$. In effect, the largest dot-products get larger weights, but in a smooth, differentiable way (unlike a hard $\arg\max$).

\subsection{Step 3: Weighted Sum}
Multiply the weight matrix by $V$:
$$\text{Attention}(Q,K,V) = A\,V,\qquad O = AV\in\mathbb{R}^{N\times d_v}.$$

Here each output row $o_i$ is a weighted sum of value vectors:
$$o_i = \sum_{j=1}^N A_{ij}\,v_j.$$

In other words, query $i$ attends to all values $v_j$ in proportion to the weight $A_{ij}$. Jurafsky \& Martin (2025) describe this procedure:

\begin{quote}
``Once we have the $QK^\top$ matrix, we can scale these scores, take the softmax, and then multiply the result by $V$ resulting in a matrix of shape $N\times d$\ldots''.
\end{quote}

Putting this together, the matrix formula for scaled dot-product attention is given in Vaswani et al. (2017) as:
$$\boxed{\text{Attention}(Q,K,V) = \text{softmax}\left(\frac{QK^\top}{\sqrt{d_k}}\right)V}$$

This compact equation encapsulates the above steps. Note that the softmax is applied independently to each query's row of scores, so each $q_i$ produces its own weight vector over all keys.

\section{Calculating Attention Weights}

More explicitly, for each query vector $q_i$ (row $i$ of $Q$) the attention weight $\alpha_{ij}$ on value $v_j$ is given by:
$$\alpha_{ij} = \frac{\exp\left(q_i\cdot k_j/\sqrt{d_k}\right)}{\sum_{j'=1}^N \exp\left(q_i\cdot k_{j'}/\sqrt{d_k}\right)}.$$

Then the output for query $i$ is 
$$o_i = \sum_{j=1}^N \alpha_{ij}\,v_j.$$ 

In matrix form this is exactly the $i$th row of $AV$. Thus the attention weight matrix $A=\text{softmax}(QK^\top/\sqrt{d_k})$ has entries $A_{ij}=\alpha_{ij}$. In summary:

\begin{itemize}
\item Compute score vector $s_i = q_i K^\top$ (dot products with all keys).
\item Scale $s_i' = s_i/\sqrt{d_k}$.
\item Normalize $\alpha_i = \text{softmax}(s_i')$, so $\alpha_i\in\mathbb{R}^N$ sums to 1.
\item Form output $o_i = \alpha_i V$.
\end{itemize}

This procedure ensures each output is a convex combination of the rows of $V$, weighted by how ``relevant'' each key is to the query.

\section{Softmax and Scaling Insights}

The softmax function is crucial because it converts raw dot-product scores into a differentiable probability distribution. By exponentiating and normalizing, softmax emphasizes the largest scores while keeping all weights positive and summing to 1. The Transformer authors note that softmax is a continuous, differentiable alternative to a hard max operation. This smoothness is essential for gradient-based optimization. Moreover, applying softmax row-wise means each query independently attends to keys.

The scaling factor $1/\sqrt{d_k}$ arises from variance considerations. Vaswani et al. show that if the components of query and key vectors are independent with variance 1, then the dot-product $q_i\cdot k_j$ has variance $d_k$. Without scaling, large $d_k$ would push $q_i\cdot k_j$ to large magnitudes, driving the softmax into regions with extremely small gradients (saturating the softmax). To avoid this, we divide by $\sqrt{d_k}$ so that the dot products have unit variance on average. In practice, this normalization keeps the softmax inputs at a scale where the exponential function is well-behaved and gradients are stable.

In summary, softmax ensures a proper probability weighting over keys, and the $\sqrt{d_k}$ scaling prevents very large or very small softmax inputs for high-dimensional vectors.

\section{Multi-Head Attention}

The Transformer improves representational power by using multi-head attention. Instead of one attention, we run $h$ parallel ``heads,'' each with its own projection of queries, keys, and values. Concretely:

\begin{itemize}
\item We choose $h$ attention heads. For head $i=1,\dots,h$, we have separate learned projections $W_i^Q,W_i^K\in\mathbb{R}^{d_{\text{model}}\times d_k}$ and $W_i^V\in\mathbb{R}^{d_{\text{model}}\times d_v}$.

\item Compute head-specific queries/keys/values: $Q_i = XW_i^Q$, $K_i = XW_i^K$, $V_i = XW_i^V$, each of size $N\times d_k$ (for $Q_i,K_i$) and $N\times d_v$ (for $V_i$).

\item Each head $i$ independently performs scaled dot-product attention:
$$\text{head}_i = \text{Attention}(Q_i,K_i,V_i) = \text{softmax}\left(\frac{Q_iK_i^\top}{\sqrt{d_k}}\right)V_i.$$
This yields $h$ output matrices $\text{head}_i\in\mathbb{R}^{N\times d_v}$.

\item Concatenate the heads along the feature dimension: 
$$H = [\text{head}_1;\,\dots;\,\text{head}_h] \in\mathbb{R}^{N\times (h\,d_v)}.$$

\item Apply a final linear projection $W^O\in\mathbb{R}^{(h\,d_v)\times d_{\text{model}}}$ to combine heads:
$$\text{MultiHead}(X) = HW^O \in \mathbb{R}^{N\times d_{\text{model}}}.$$
\end{itemize}

Jurafsky \& Martin summarize this as: ``we linearly project the queries, keys and values $h$ times with different learned projections\ldots perform the attention function in parallel\ldots concatenate and once again project, resulting in the final values''. In formula form, Vaswani et al. give:

\begin{align}
\text{head}_i &= \text{Attention}(XW_i^Q, XW_i^K, XW_i^V) \quad (i= 1, \ldots, h)\\
\text{MultiHead}(X) &= [\text{head}_1; \ldots; \text{head}_h] W^O
\end{align}

More compactly, with $\oplus$ denoting concatenation:
$$\boxed{\text{MultiHead}(X) = (\text{head}_1 \oplus \text{head}_2 \oplus \cdots \oplus \text{head}_h)\,W^O}$$

Multi-head attention allows the model to attend to information from different representation subspaces; each head can focus on different patterns or relations in the input. After concatenation and projection, the output has the same shape as a single-head output ($N\times d_{\text{model}}$), ready to be fed into the next layer of the transformer.

\section{References}

The above formulations follow Vaswani et al. (2017) and Jurafsky \& Martin (2025). The use of softmax and scaling is discussed in both sources, with Vaswani et al. explaining the scaling rationale and providing the standard softmax formulation.

\begin{enumerate}[label={[\arabic*]}]
\item Jurafsky, D., \& Martin, J. H. (2025). \textit{Speech and Language Processing} (3rd ed.). \url{https://web.stanford.edu/~jurafsky/slp3/8.pdf}

\item Vaswani, A., Shazeer, N., Parmar, N., Uszkoreit, J., Jones, L., Gomez, A. N., Kaiser, L., \& Polosukhin, I. (2017). Attention is all you need. \textit{Advances in Neural Information Processing Systems}, 30. \url{https://papers.neurips.cc/paper/7181-attention-is-all-you-need.pdf}

\item Mody, J. An Intuition for Attention. \url{https://jaykmody.com/blog/attention-intuition/}

\item Learning-Deep-Learning. Transformer: Attention Is All You Need. \url{https://patrick-llgc.github.io/Learning-Deep-Learning/paper_notes/transformer.html}
\end{enumerate}

\end{document}
